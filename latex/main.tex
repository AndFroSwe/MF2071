% TODO: Q2A
% TODO: Q2B
% TODO: Q2C
% TODO: Q2D
% TODO: Q2E
% TODO: Q2F

\documentclass[a4paper, 12pt]{article}

\usepackage[english]{babel}

\begin{document}
%%%%%%%%%% Title %%%%%%%%%%%%%%%%%%%%
\title{MF2071 Home Exam}
\author{Andreas Fr\"{o}derberg}
\maketitle
%%%%%%%%%%%%%%%%%%%%%%%%%%%%%%%%%%%%%
% Q1A: Describe your field of studies and its scientific context. Break out an information need that you find
%relevant. It is good if this information need is for your thesis or a similar report. Try to be as specific as
%possible.
\section*{Question 1}
\subsection*{A}
My field of studies is Mechatronics, built upon a bachelors degree in Mechanical
engineering. Mechatronics is often described as the unity of control systems,
computers, mechanical systems and electronic systems. Because of the wide spread
of subjects a mechatronics engineer has knowledge of, the engineer might be 
considered a jack of all trades, master of none. The strength of this focus on 
being not focused is that the mechatronic engineer is a specialist on integrating
all the subjects into one final design. 

This home exam, and the work in this course, will be based on the master thesis
job I will do at AVL Powertrain Engineering. AVL is the worlds largest independent
company focusing on development and testing of powertrains and drivelines and is
based in Graz, Austria. This thesis will be done for the AVL branch in 
S\"{o}dert\"{a}lje, Sweden. They want an autonomous car that can be used for
testing and research and as a part of that project, they want to investigate
which localization and mapping (SLAM) algorithm that is suitable to use 
from a cost/performance perspective. More specifically, different SLAM algorithms
have different performance and different demands on sensors and hardware 
calculation capacity. Also, some algotihms perform well in simulation but not in the 
field. 

There is both a technical aspect of the information need, the investigation of
different SLAM algorithms and their pros and cons, and a subjective aspect in 
the choice of what makes a ``good'' algorithm for this specific purpose. For this 
report, the technical aspect is chosen to be expanded upon since the largest part 
of the thesis will be the gathering and investigation of different algorithms.
% Q1B: Use the problem from question 1A. Can you find different angles to approach this problem/
%information need? Describe why this is an advanced information need. Also describe which sources
%you need to solve the information need. (Hint: If you can’t find any reasons to why you would need
%information from several sources the information need is not advanced enough for this course).
\subsection*{B}
Here, the answer is given for a general algorithm that needs to be implemented on some
kind of hardware. It is feasable to consider SLAM along with any kind of algorithm that
needs to convert physical input to some kind of concatenated information since 
many of these algorithms share the same broad terms of pros and cons. 
An algorithm can be viewed and assessed from different perspectives depending. 
From a computing standpoint it is interesting to know how much computing power is 
needed to calculate the output of the algorithm and how the computing time increases
with increased accuracy. The result of this will in turn be of interest from a
hardware standpoint as the processor must be powerful enough to handle the computations.
If power consumtion is relevant, then the complexity will be of even more interest.
Further, an algorithm can be judged based on the number of inputs it needs. More
inputs will result in the need of more sensors which increases the cost of the 
implementation. Not only that, but the kind of input needed is also of importance 
since different kinds of sensors have different costs associated with them.
Lastly, SLAM algorithms have different performance in real life and in simulation.
This leads to the search question: {\it Which SLAM algorithms in an autonomous car are 
suitable from a hardware cost perspective?}

To find relevant information about SLAM, several sources are needed. To find 
which algorithms are confined within the SLAM umbrella and to get a general
overview of the subject, popular information sources such as news articles can 
be browsed, as well as commercial sources. These information sources can 
give a broad overview but will not be regarded as primary sources. Conference 
proceedings can also be used to get an overview. Using the search
terms gathered from the popular information sources, scholarly sources can be 
searched for primary information sources. These might be research papers or 
scientific books or journals written by researchers in the relevant fields. It 
is here that the most detailed search query iteration will be done. 

This problem is advanced since it covers several different disciplines of science
and that research is being done in the subject of localization and mapping of
autonomous vehicles right now. 



\end{document}
