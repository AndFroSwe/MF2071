This project focuses on comparing different methods for a vehicle to localize
itself in the world. The comparison will be made using experimental design
methodology with a post-positivistic world view. This means that the data should
be collected and analyzed objectively, proving or disproving one or several
hypothesis. To be able to make an objective assessment of the different methods,
extra weight will be given to sourcing well defined assessment criterea and
requirements from the project owner, AVL. These criterea will then be used to
create a testing framework for the different methods, reducing the risk for
result bias.

Using multiple sources will help in getting an as complete picture of the method
as possible. When looking for preliminary sources, a database that covers many
subjects is queried. For this report, Scopus was used for finding preliminary
sources. Wide search terms are used to capture as many sources as possible and a
wide variety of sources are allowed. For example, conference proceedings can
give good insight into what is happening within a subject but may not be
suitable to use as a primary source. 

The preliminary data is then used to refine the search in a more targeted search
engine. These databases only gather documents from specific areas of research.
Being a project focused in electrical engineering, mechatronics and computer
science, the IEEE Xplore database is used for in depth searching.  

