In hindsight of the defined refinement method
there is a huge gain in iterating the search wisely. It is likely that the
results from one of the searches will contain keywords used when describing
other methods. It can also be possible to discard some search results based
solely on them containg specific keywords. Using multiple sources will help in
getting an as complete picture of the method as possible. When looking for
preliminary sources, a database that covers many subjects is queried. For this
report, Scopus was used for finding preliminary sources. Wide search terms are
used to capture as many sources as possible and a wide variety of sources are
allowed. For example, conference proceedings can give good insight into what is
happening within a subject but may not be suitable to use as a primary source. 

The preliminary data is then used to refine the search in a more targeted search
engine. These databases only gather documents from specific areas of research.
Being a project focused in electrical engineering, mechatronics and computer
science, the IEEE Xplore database is used for in depth searching.  

