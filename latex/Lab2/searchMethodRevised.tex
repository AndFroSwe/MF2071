This project focuses on comparing different methods for a vehicle to localize
itself in the world. The comparison will be made using experimental design
methodology with a post-positivistic world view. This means that the data should
be collected and analyzed objectively, proving or disproving one or several
hypothesis\cite{creswell}. To be able to make an objective assessment of the different methods,
extra weight will be given to sourcing well defined assessment criteria and
requirements from the project owner, AVL.\@ These criteria will then be used to
create a testing framework for the different methods, reducing the risk for
result bias.

Using multiple sources will help in getting an as complete picture of the method
as possible. When looking for preliminary sources, a database that covers many
subjects is queried. For this report, Scopus was used for finding preliminary
sources. Wide search terms are used to capture as many sources as possible and a
wide variety of sources are allowed. For example, conference proceedings can
give good insight into what is happening within a subject but may not be
suitable to use as a primary source. 

The preliminary data is then used to refine the search in a more targeted search
engine. These databases only gather documents from specific areas of research.
Being a project focused in electrical engineering, mechatronics and computer
science, the IEEE Xplore database is used for in depth searching.  

While the subject of self controlling cars has many ethical concerns attached to
it, most are about what choice to make in a situation like the~\textit{trolley
problem}\footnote{The
trolley problem is a philosophical question about choosing which life/lives to
save/sacrifice in a two-choice situation.}. This does not directly
affect the localization and mapping part of the car control, though it can be
argued that a better situational awareness can be a way to avoid fail state
situations. 
% This is the revised part // AF
As an example, consider that an autonomous car comes to a crossing and a 
pedestrian walks out in front of the car. The car then has to make a decision:
should it just break and risk hitting the pedestrian or should it veer to the side
and potentially crash but save the pedestrian? If the car has a good knowledge
about its surroundings (i.e.\ an efficient SLAM algorithm), it can make a more 
informed decision. Also, should the system have a good map of the area, there
are possibilities that the car can be programmed to avoid busy crossing and 
will therefore also greatly reduce the risk of being caught in that kind of 
situations at all.

SLAM does not in any large way affect the fuel consumption of the car so it is
not affected by ecologic sustainability concerns. It does however directly
influence of an autonomous vehicle. It is, in fact, critical to the
functionality of the product and therefore there are both social and economic
sustainability attached to finding good SLAM algorithms. 
