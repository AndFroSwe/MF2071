Simultaneous Location And Mapping (SLAM) is the ability for a autonomous
device or vehicle to start at an unknown location, in an unknown environment,
and incrementally build a map of this environment. This map is simultaneously
used to compute an absolute vehicle location~\cite{938381}. The project owner,
AVL~\footnote{AVL is the world's largest independent company for the
development of powertrain systems with internal combustion engines as well as
instrumentation and test systems.}, want to investigate which localization and
mapping (SLAM) algorithm that is suitable to use from a cost/performance
perspective. More specifically, different SLAM algorithms have different
performance and different demands on sensors and hardware calculation capacity.
Also, some algorithms perform well in simulation but not in the field.  There is
both a technical aspect of the information need, the investigation of different
SLAM algorithms and their pros and cons, and a subjective aspect in the choice
of what makes a suitable algorithm for this specific purpose. For this report,
the technical aspect is chosen to be expanded upon since the largest part of
the thesis will be the gathering and investigation of different algorithms.
